\begin{definition}
    Sei $f: V\to \K$ eine lineare Abbildung auf $V$. Dann heißt $f$ Linearform.
\end{definition}
\begin{definition}
    Die Menge aller linearen Abbildungen $f:V\to\K$ auf $V$ heißt Dualraum von V
    und wird mit $V^*$ notiert. Der Dualraum $V^*$ von $V$ ist mit den folgenden
    Operationen ein Vektorraum: $f,g \in V^*, \lambda \in \K,$ und $v\in V$
    \begin{align*}
        (f+g)(v) &\coloneqq f(v) + g(v)\\
        (\lambda f)(v) &\coloneqq \lambda(f(v))
    \end{align*}
\end{definition}
\begin{theorem}
    Sei $V$ ein endlich dimensionaler Vektorraum über einen Körper $\K$. Es ist
    \[
        \dim V = \dim V^*
    \]
\end{theorem}
\begin{proof}
    Später
\end{proof}
\begin{definition}
    Sei $V$ ein endlich dimensionaler Vektorraum über einen Körper $\K$, und $B
    = (v_1, \dots, v_n)$ eine Basis von $V$. Dann bezeichnen wir mit $v_i^* \in
    V^*$ die Linearform \[
        v_i^*\left(\dsum_{j=1}^n \lambda_i v_j\right) = \lambda_i, \midspace
         \lambda_i \in \K, 1\leq i \leq n\]
    und $(v_1^*, \dots, v_n^*)$ heißt duale Basis zu $B$.
\end{definition}

