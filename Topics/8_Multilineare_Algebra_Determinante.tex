\newcommand{\vol}[1]{\ensuremath{\txt{vol}\left(#1\right)}}
\begin{definition}[Determinante $1\times 1$ und $2\times 2$ Matrizen]
    \begin{itemize}
        \item Für $a \in \calM_1 (\mathbb{K})$ gilt $\det a = a$.
        \item Sei $a \coloneqq \matrx{a&b\\c&d}$ Es ist $\det a = ad-bc$. 
    \end{itemize}
\end{definition}
\begin{definition}[Laplace Entwicklung]
    Ist $n\geq 2$ und $A\in \calM_(\K)$, so ist \[
        \det A = \sum_{j=1}^n (-1)^{1+j} a_{1j}\det A_{1j}\]    
\end{definition}
\begin{definition}[Sarrus-Regel]
    Für $A\in \calM_3$ gilt \[
        \det A = a_{11}a_{22}a_{33} - a_{12}a_{21}a_{33} - a_{13}a_{22}a_{31} + 
        a_{12}a_{23}a_{31} + a_{13}a_{21}a_{32} - a_{11}a_{23}a_{32}\]
\end{definition}
\begin{definition}[Determinantenabbildung]
    Eine Determinantenabbildung auf $\calMn(\K)$ ist eine Abbildung $D:\calMn(\K)
    \to \K$ so, dass\begin{enumerate}
        \item $\forall i \in \set{1,\dots, n}$ ist $D$ linear in der i-ten Zeile,
        also \[
            D\left(\matrx{a_1\\\vdots\\a_i+\lambda b_i\\\vdots\\v_n}\right) = 
            D\left(\matrx{a_1\\\vdots\\a_i\\\vdots\\a_n}\right) + \lambda D\left(\matrx{
                a_1\\\vdots\\b_i\\\vdots\\a_n
            }\right)\]
        \item Hat $A\in \calMn(\K)$ zwei gleiche Zahlen, so ist $D(A) = 0$ (d.h.
        $D$ ist alternierend) 
        \item $D(I_n) = 1$
    \end{enumerate}    
\end{definition}
\begin{theorem}\hfill\break
    \begin{enumerate}[(i)]
        \item \[
            \det A = \sum_{j=1}^n (-1)^{1+j}a_{1j}\det A_{1j}\]
        \item \[
            \det A = \sum_{\sigma\in S_n}\txt{sgn}(\sigma)\prod_{i=1}^n a_{1\sigma}(i)\]
            mit $S_n :$ Permutationen von $\set{1,2,\dots,n}$ und sgn$(\sigma):$ 
            Fehltstände oder Signum von $\sigma$
        \item \[\abs{\det\matrx{a&b\\c&d}} = \abs{ad-bc}\equiv\txt{Fläche des 
        von $\matrx{a\\c}, \matrx{b\\c}$ aufgespannten Parallelogram.}\]
        \item Determinantenabbildung $
            D:\calMn(\K) \to \K:$\begin{enumerate}
                \item Linear in jeder Zeile
                \item Hat die Matrix $A\in \calMn(\K)$ zwei gleiche Zeilen, so ist 
                $\det A = 0$
                \item $\det I_n = n$
            \end{enumerate}
    \end{enumerate}
\end{theorem}
\begin{remark}
    Man sucht in der Definition der Determinante die \glqq Eigenschaften des Volumens\grqq
    in jeder Dimension, z.B. vol$(\lambda K) = \lambda^2$vol$(K), K\subset\R^2$.
    \newline\textbf{Geometrische Motivation} \begin{enumerate}
        \item Volumen ist Null wenn die 3 Vektoren auf einer Ebene liegen
        \item Volumen des Wuerfels mit Kantenlänge 1 ist 1 
        \item Verdoppelt man $v_1,v_2,v_3$ so wird das Volumen 8 mal größer \[
            \vol{\lambda_1v_1, \lambda_2v_2, \lambda_3v_3} = \lambda_1\lambda_2
            \lambda_3\vol{v_1,v_2,v_3}\]
    \end{enumerate}
    Volumen von Vektoren ist als Volumen des von $v_1,v_2,v_3$ aufgespannten 
    Parallelepiped zu verstehen. Vektoren in einer Determinante sind in Zeilen zu 
    verstehen.
    \newline\textbf{Die Determinante erfüllt:}\begin{enumerate}
        \item Determinante ist Null, wenn die 3 Vektoren l.a. sind
        \item Die Determinante der Einheitsmatrix ist 1
        \item $\det \matrx{\lambda_1x_1\\\lambda_2x_2\\\lambda_3x_3} = \lambda_1
        \lambda_2\lambda_3 \det\matrx{x_1\\x_2\\x_3}$
    \end{enumerate}
    Die Determinante erfüllt also all die Eigenschaften die wir geometrisch haben
    wollen.
\end{remark}
\begin{definition}
Sei $A \in \calMn(\mathbb{K})$. TFAE
\begin{enumerate}
    \item det $A \neq 0$
    \item $A$ ist invertierbar
    \item Es gibt $B \in \calMn(\mathbb{K})$ mit $AB = I_n$
    \item Rang$A=n$
    \item $T_A: \mathbb{K}^n \to \mathbb{K}^n, \quad x\mapsto Ax$ ist bijektiv
    \item $T_A: \mathbb{K}^n \to \mathbb{K}^n, \quad x\mapsto Ax$ ist surjektiv
    \item $T_A: \mathbb{K}^n \to \mathbb{K}^n, \quad x\mapsto Ax$ ist injetiv
    \item Kern$T_A=0$
    \item Es gibt Elementarmatrizen $R_1,...,R_k$ so, dass $A=R_k...R_1\cdot I_n$
    \item Mittels elementarer Zeilenumformungen bekommt man aus $A$ die Matrix $I_n$
    \end{enumerate}
\end{definition}

\begin{theorem}
Sei $D:\calMn(\mathbb{K}) \to \mathbb{K}$ eine Determinantenabbildung und sei \\ $A \in \calMn(\mathbb{K})$. Dann ist $A$ invertierbar genau dann, wenn $D(A) \neq 0$
\end{theorem}

\begin{theorem}
Falls eine Determinantenfunktion det$:\calMn(\mathbb{K}) \to \mathbb{K}$ existiert, so ist diese eindeutug bestimme. Also, es gibt höchstens eine Determinantenabbildung.
\end{theorem}

\begin{theorem}
Es gibt genau eine Determinantenabbildung auf $\mathbb{K}$ , nämlich
\begin{align*}
    \det A=\sum_{\sigma\in S_n}sgn(\sigma) \prod_{i=1}^n a_{i\sigma(i)}.
\end{align*}
Es reicht zu zeigen, dass det eine Determinantenabbildung ist, nach Satz (2 vorher)
\end{theorem}

\begin{theorem}
Sei $A\in \calMn(\mathbb{K})$. Dann gilt 
\begin{align*}
    \det A=\det A^T
\end{align*}
\end{theorem}

\begin{theorem}
Seien $A,B\in \calMn(\mathbb{K})$. Es gilt
\begin{align*}
    \det(AB)=\det A \det B.
\end{align*}
\end{theorem}

\begin{theorem}
Die menge der invertierbaren Matzizen
\begin{align*}
    GL(n,\mathbb{K}) \coloneqq \{A\in \calMn(\mathbb{K}) : \det A \neq 0\}
\end{align*}
ist eine Gruppe bzgl. der Matrixmultiplikation
\end{theorem}

\begin{theorem}
Sei $n\geq 2$ und $A=(a_{ij}) \in \calMn(\mathbb{K})$. Es gilt:
\begin{enumerate}
    \item Entwicklung nach der $i-$ten Zeile
    \begin{align*}
        \det A = \sum_{j=1}^n (-1)^{\textcolor{red}{i}+j}a_{\textcolor{red}{i}}\det A_{\textcolor{red}{i}j}
    \end{align*}
    \item Entwicklung nach der $j-$ten Spalte
    \begin{align*}
        \det A = \sum_{j=1}^n (-1)^{i+\textcolor{red}{j}}a_{i\textcolor{red}{j}}\det A_{i\textcolor{red}{j}}
    \end{align*}
\end{enumerate}
\end{theorem}