\newcommand{\stdscp}{\ensuremath{\scp{\cdot,\cdot}}}
\newcommand{\fuer}{\ensuremath{\text{ für }}}
\newcommand{\KeRoC}{(\K = \R \smallspace \lor \smallspace\K = \C)}
\begin{definition}
    Sei $V$ ein $\R$ Vektorraum. Eine positiv definite symmetrische Bilinearform
    auf $V$ heißt (euklidisches) Skalarprodukt. Ist $\stdscp$ ein 
    (euklidisches) Skalarprodukt auf den reellen Vektorraum $V$, so heißt $(V, 
    \stdscp)$ ein Euklidischer Vektorraum.
\end{definition}
\begin{example}
    Auf $\R^n$ ist $\scp{x,y} = x^T y$ ein euklidisches Skalarprodukt. Wir 
    nennen es Standard Skalarprodukt und es ist das Standardbeispiel eines 
    euklidischen (endlich-dimensionalen) Vektorraum.
\end{example}
\begin{definition}
    Sei $V$ ein komplexer Vektorraum und sei $\stdscp: V\times V\to
    \C$ eine Abbildung mit den folgenden Eigenschaften:\begin{enumerate}
        \item $\stdscp$ ist linear in der ersten Variable, d.h. \[
            \forall v_1, v_1', v_2 \in V \smallspace\forall \lambda \in \C: 
            \scp{\lambda v_1 + v_1', v_2} = \lambda\scp{v_1, v_2} + \scp{v_1',
             v_2}\]
        \item $\stdscp$ ist sesquilinear in der zweiten Variable, d.h
            \[
                \forall v_1, v_2, v_2' \in V \smallspace \forall \lambda \in \C
                : \smallspace \scp{v_1, \lambda v_2 + v_2'} = \compconj{\lambda}
                \scp{v_1, v_2} + \scp{v_1, v_2'}\]
    \end{enumerate}
    Dann heißt $\stdscp$ eine Sesquilinearform.
\end{definition}
\begin{definition}
    Eine Sesquilinearform auf den komplexen Vektorraum $V$ heißt\begin{itemize}
        \item hermitesch wenn zusätzlich gilt\[
            \scp{v_1, v_2} = \compconj{\scp{v_1,v_2}}\]
        \item positiv definit wenn \[
            \forall 0 \neq v \in V: \smallspace \scp{v,v} > 0\]
    \end{itemize}
\end{definition}
\begin{definition}
    Sei $V$ ein $\C$-Vektorraum. Eine positiv definite hermitesche 
    Sesquilinearform $\stdscp:V\times V\to \C$ auf $V$ heißt unitäres
    Skalarprodukt oder inneres Produkt. Ist $\stdscp$ ein inneres Produkt auf 
    den komplexen Vektorraum $V$, so heißt $(V, \stdscp)$ ein unitärer 
    Vektorraum.
\end{definition}
\begin{definition}
    Sei $(a_{ij}) = A\in \calM_{n\times m}(\C)$. Dann ist $A^H = (\compconj{
        a_{ij}})^T$ die hermitesch Transponierte von $A$. Sind alle Einträge von
        $A \in \calM_{n\times m}(\C)$ reell, so ist $A^H = A^T$.
\end{definition}
\begin{example}
    Auf $\C^n$ ist $\scp{x,y}= y^H x= \dsum_{i=1}^n \compconj{y_i}x_i$ ein 
    inneres Produkt. Wir nennen es Standard inneres Produkt und es ist das 
    Standardbeispiel eines unitären (endlich-dimensionalen) Vektorraums.
\end{example}
\begin{remark}
    \hfill
    \begin{enumerate}
        \item Ist $\stdscp$ ein inneres Produkt ($\K = \C$) oder ein 
        Skalarprodukt ($\K = \R$), so ist $\forall v \in V: \scp{v,v} \geq 0$ 
        \item Ein Unterraum eines euklidischen oder unitären Vektorraums ist 
        wieder ein euklidischer bzw. unitärer Vektorraum.
        \item In der zweiten Variable ist ein Skalarprodukt linear. In der 
        zweiten Variable ist ein inneres Produkt sesquilinear. In beiden Fällen
        gilt\[
            \scp{u, \alpha v + \beta w} = \compconj{\alpha}\scp{u,v} + 
            \compconj{\beta}\scp{u,w}\]
        \item $\scp{v,0}= 0 = \scp{0,v}$
    \end{enumerate}
\end{remark}
\begin{theorem}
    Sei $V$ ein euklidischer oder ein unitärer Vektorraum. Dann gilt\[
        \forall x,y \in V: \smallspace \scp{x,y}\compconj{\scp{x,y}} \leq
        \scp{x,x}\scp{y,y}\]
        Die Gleichheit gilt genau dann wenn $x,y$ linear abhängig sind.
\end{theorem}
\begin{proof}
    Später
\end{proof}
\begin{theorem}
    Sei $V$ ein euklidischer ($\K = \R$) oder ein unitärer ($\K = \C$) Vektorraum
    Dann gilt \begin{enumerate}
        \item $\forall x,y \in V: \scp{\lambda x, \lambda x} = \lambda \compconj
        {\lambda}\scp{x,x}$
        \item $\scp{x+y, x+y} \leq \left(\scp{x,x}^\frac{1}{2} + \scp{y,y}^\frac
        {1}{2}\right)^2$, mit Gleichheit gdw. $x = \alpha y \lor y = \alpha x,
        \smallspace \alpha \geq 0, \alpha \in \R$
    \end{enumerate}
\end{theorem}

\begin{definition}
    Sei $V$ ein euklidischer oder unitärer Vektorraum. Für $v\in V$ heißt \[
        \dabs{v} = \left(\scp{v,v}\right)^\frac{1}{2}\]
    die Länge des Vektors $v$.
\end{definition}

\begin{remark}
    Sei $V$ ein euklidischer oder unitärer Vektorraum. Dann ist \[
        \scp{v,v}\geq 0 \smallspace \scp{v,v} > 0, \midspace v\neq 0\]
    Somit ist $\scp{v,v} = 0 \iff v = 0$ 
\end{remark}

\begin{definition}
    Sei $V$ ein $\K$ Vektorraum mit $\K = R \lor \K = \C$. Eine Abbildung\[
        \dabs{\cdot}: V\to \R, \midspace v\mapsto \dabs{v}\]
    heißt Norm auf $V$ wenn für alle $v,w \in V, \smallspace \lambda \in \K$ 
    gilt:\begin{enumerate}[(a)]
        \item $\forall v \in V \dabs{v} \geq 0 \smallspace \land\smallspace 
        \dabs{v} = 0 \iff v=0$
        \item $\forall \lambda in \K \smallspace \forall v \in V: \dabs{\lambda v}
         = \abs{\lambda} \dabs{v}$
        \item $\dabs{x+y} \leq \dabs{x} + \dabs{y}$
    \end{enumerate}
\end{definition}

\begin{definition}
    Sei $V$ ein euklidischer oder unitärer Vektorraum. Dann ist \[
        x \mapsto \scp{x,x}^\frac{1}{2}\]
    eine Norm auf $V$.
\end{definition}
\begin{remark}
    Nicht jede Norm stammt aus einem Skalarprodukt bzw einem inneren Produkt!
\end{remark}
\begin{definition}
    Sei $V$ ein $\K$ Vektorraums, $\K = \R \lor \K = \C$ mit dem Skalarprodukt
    (oder inneren Produkt) $\stdscp$. \begin{itemize}
        \item $x,y \in V$ mit $\scp{x,y} = 0$ heißen orthogonal zu einander. Wir
        schreiben $x\perp y$.
        \item Zwei Teilmengen $M_1,M_2 \subset V$ heißen zu einander orthogonal,
        wenn \[
            \forall x_1 \in M_1 \forall x_2 \in M_2: x_1 \perp x_2\]
        Wir schreiben $M_1 \perp M_2$
        \item Die Teilmenge $S\subset V$ heißt Orthonormalsystem, wenn \[
            \forall x,y\in S: \scp{x,y} = \delta_{xy}, \text{ d.h. } \scp{x,y} =
            \begin{cases}
            1 & \fuer x=y\\
            0 & \fuer x\neq y    
            \end{cases}\]
        \item Ist $\emptyset \neq M \subset V$ eine nichtleere Teilmenge, dann 
        heißt \[
            M^\perp \coloneqq \set{y \in V\mid \forall x \in M: \scp{x,y} = 0}\]
        der zu $M$ orthogonale Unterraum.
    \end{itemize}
\end{definition}
\begin{theorem}
    Sei $V$ ein $\K$-Vektorraum $\KeRoC$ mit dem Skalarprodukt \stdscp
    \begin{enumerate}
        \item Ist $S$ ein Orthonormalsystem, dann ist $S$ linear unabhängig 
        \item Ist $B = \set{b_1, \dots, b_n, \dots}$ eine abzählbare linear 
        unabhängige Teilmenge von $V$, dann gibt es genau ein Orthonormalsystem
        $S = \set{d_1, d_2, \dots}$ mit \[
            d_k = \sum_{j \leq k} \alpha_{jk}b_j, \midspace \alpha_{jk} \in \K,
            \smallspace j \leq k,\smallspace \alpha_{kk} > 0 \]
        Insbesondere gilt \[
        \forall I \in \N\lin{b_1, b_2, \dots, b_I} = \lin{d_1, d_2, \dots, d_I}\]
        Gramm-Schmidtsche Orthogonalisierung.
    \end{enumerate}
\end{theorem}
\begin{proof}\hfill
    \begin{enumerate}
        \item 
        Sei $S\subset V$ ein Orthonormalsystem, sei $m\in \N$ und seien $v_1, \dots,
        v_m \in S$, $\lambda_1, \dots, \lambda_m \in \K$ mit \[
            \dsum_{j=1}^m \lambda_j v_j = 0\]
        Für $1\leq k \leq m$ gilt \[
            0 = \scp{0, v_k} = \scp{\dsum_{j=1}^m \lambda_j v_j, v_k} = 
            \dsum_{j=1}^m \lambda_j \scp{v_j, v_k} \underset{(*)}{\underbrace{=}}
            \lambda_k\]
            (*): Da $S$ ein Orthonormalsystem ist gilt, dass $\forall x,y \in S:
            \smallspace \scp{x,y} = \begin{cases}
                1 & \fuer x=y\\
                0 & \fuer x \neq y
            \end{cases}$. Somit ist $S$ linear unabhängig.
        \item Es gelte (2), d.h.:\[
            d_k = \dsum_{j\leq k} \alpha_{jk}b_k, \midspace \alpha_{jk}\in \K, 
            \smallspace j\leq k,\smallspace \alpha_{kk} > 0 \]
            so ist \[
                \lin{d_1, \dots, d_I} \subset \lin{b_1, \dots, b_I}\]
            
    \end{enumerate}
\end{proof}